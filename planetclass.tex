% ****** Start of file apssamp.tex ******
%
%   This file is part of the APS files in the REVTeX 4.1 distribution.
%   Version 4.1r of REVTeX, August 2010
%
%   Copyright (c) 2009, 2010 The American Physical Society.
%
%   See the REVTeX 4 README file for restrictions and more information.
%
% TeX'ing this file requires that you have AMS-LaTeX 2.0 installed
% as well as the rest of the prerequisites for REVTeX 4.1
%
% See the REVTeX 4 README file
% It also requires running BibTeX. The commands are as follows:
%
%  1)  latex apssamp.tex
%  2)  bibtex apssamp
%  3)  latex apssamp.tex
%  4)  latex apssamp.tex
%
\documentclass[%
 reprint,
%superscriptaddress,
%groupedaddress,
%unsortedaddress,
%runinaddress,
%frontmatterverbose, 
%preprint,
%showpacs,preprintnumbers,
%nofootinbib,
%nobibnotes,
%bibnotes,
 amsmath,amssymb,
 aps,
%pra,
%prb,
%rmp,
%prstab,
%prstper,
%floatfix,
]{revtex4-1}

\usepackage{graphicx}% Include figure files
\usepackage{dcolumn}% Align table columns on decimal point
\usepackage{bm}% bold math
%\usepackage{hyperref}% add hypertext capabilities
%\usepackage[mathlines]{lineno}% Enable numbering of text and display math
%\linenumbers\relax % Commence numbering lines

%\usepackage[showframe,%Uncomment any one of the following lines to test 
%%scale=0.7, marginratio={1:1, 2:3}, ignoreall,% default settings
%%text={7in,10in},centering,
%%margin=1.5in,
%%total={6.5in,8.75in}, top=1.2in, left=0.9in, includefoot,
%%height=10in,a5paper,hmargin={3cm,0.8in},
%]{geometry}

\begin{document}

\preprint{}

\title{Earth is a class A planet}% Force line breaks with \\
\thanks{Beware of the date.}%

\author{Interplanetary Immigration Center Agents}
 \email{admin@interimm.org}
\affiliation{%
 Interplanetary Immigration Center\\
 Section 9, Mars.
}%

\collaboration{Interplanetary Immigration Center Exoplanet Collaboration}%\noaffiliation

\date{\today}% It is always \today, today,
             %  but any date may be explicitly specified

\begin{abstract}
Interplanetary Immigration Center Exoplanet Collaboration
\begin{description}
\item[Access] CC BY-SA
\item[Structure]
The number of exoplanets keeps increasing in the past years. As we all known, exoplanets have diversities in varies aspects. Work has been done to specify a GPA, or Grade Point Average, to exoplanets with regard to their values of human immigration on to the planet. Different subjects, including water, temperature and atmosphere, are taken into consideration, where each subject is graded based on the quality of life on that planet.
\end{description}
\end{abstract}

\pacs{Valid PACS appear here}% PACS, the Physics and Astronomy
                             % Classification Scheme.
%\keywords{Suggested keywords}%Use showkeys class option if keyword
                              %display desired
\maketitle

%\tableofcontents

\section{Plan of the paper: should be removed from the formal draft}

Motivation:

\begin{enumerate}
  \item For use of categorization
  \item A reference value
  \item Diversity of exoplanets makes it hard to use a single score for the value so we came up with a GPA method which serves the purpose well and provides a lot of flexibility. 
\end{enumerate}


What to bring in?

\begin{enumerate}
\item Overall idea
\item How to grade them subject by subject
\item Examples and validation
\end{enumerate}

How to extend the method?

Write about the extension of the method using real data and more.





\section{\label{sec:level1}Exoplanet as home}



\paragraph{Human Survival on Earth}



\end{document}
%
% ****** End of file apssamp.tex ******
